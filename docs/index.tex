%%
%% Automatically generated file from DocOnce source
%% (https://github.com/hplgit/doconce/)
%%

% #define PREAMBLE

% #ifdef PREAMBLE
%-------------------- begin preamble ----------------------

\documentclass[%
oneside,                 % oneside: electronic viewing, twoside: printing
final,                   % draft: marks overfull hboxes, figures with paths
10pt]{article}

\listfiles               %  print all files needed to compile this document

\usepackage{relsize,makeidx,color,setspace,amsmath,amsfonts,amssymb}
\usepackage[table]{xcolor}
\usepackage{bm,ltablex,microtype}

\usepackage[pdftex]{graphicx}
\usepackage{fancyvrb}

\usepackage[T1]{fontenc}
%\usepackage[latin1]{inputenc}
\usepackage{ucs}
\usepackage[utf8x]{inputenc}

% Set palatino as the default font family:
\usepackage[sc]{mathpazo}    % Palatino fonts
\linespread{1.05}            % Palatino needs extra line spread to look nice

\usepackage{lmodern}         % Latin Modern fonts derived from Computer Modern

% Hyperlinks in PDF:
\definecolor{linkcolor}{rgb}{0,0,0.4}
\usepackage{hyperref}
\hypersetup{
    breaklinks=true,
    colorlinks=true,
    linkcolor=linkcolor,
    urlcolor=linkcolor,
    citecolor=black,
    filecolor=black,
    %filecolor=blue,
    pdfmenubar=true,
    pdftoolbar=true,
    bookmarksdepth=3   % Uncomment (and tweak) for PDF bookmarks with more levels than the TOC
    }
%\hyperbaseurl{}   % hyperlinks are relative to this root

\setcounter{tocdepth}{2}  % levels in table of contents

% --- fancyhdr package for fancy headers ---
\usepackage{fancyhdr}
\fancyhf{} % sets both header and footer to nothing
\renewcommand{\headrulewidth}{1pt}
% section name to the left (L) and page number to the right (R)
% on even (E) pages, the other way around on odd pages
% (switch twoside to onside in documentclass to just have odd pages)
\fancyhead[LE,RO]{\rightmark} % section
\fancyhead[RE,LO]{\thepage}
% Ensure copyright on titlepage (article style) and chapter pages (book style)
\fancypagestyle{plain}{
  \fancyhf{}
  \fancyfoot[C]{{\footnotesize \copyright\ 2017, Rico A.R. Picone. Released under CC Attribution 4.0 license}}
%  \renewcommand{\footrulewidth}{0mm}
  \renewcommand{\headrulewidth}{0mm}
}
% Ensure copyright on titlepages with \thispagestyle{empty}
\fancypagestyle{empty}{
  \fancyhf{}
  \fancyfoot[C]{{\footnotesize \copyright\ 2017, Rico A.R. Picone. Released under CC Attribution 4.0 license}}
  \renewcommand{\footrulewidth}{0mm}
  \renewcommand{\headrulewidth}{0mm}
}

\pagestyle{fancy}


% prevent orhpans and widows
\clubpenalty = 10000
\widowpenalty = 10000

% http://www.ctex.org/documents/packages/layout/titlesec.pdf
\usepackage{titlesec}  % needed for colored section headings
%\usepackage[compact]{titlesec}  % reduce the spacing around section headings


% --- color every two table rows ---
\let\oldtabular\tabular
\let\endoldtabular\endtabular
\definecolor{appleblue}{rgb}{0.93,0.95,1.0}  % Apple blue
\renewenvironment{tabular}{\rowcolors{2}{white}{appleblue}%
\oldtabular}{\endoldtabular}


% --- end of standard preamble for documents ---

% Custom LaTeX stuff in the preamble [from latex_preamble.tex]

% fonts
\usepackage[scaled=.92]{helvet}
%\usepackage[urw-garamond]{mathdesign}
%\usepackage[osf,sc]{mathpazo}
\usepackage{palatino}
\usepackage[T1]{fontenc}
\usepackage{eucal}  % nice mathcal fonts
\usepackage[small,euler-digits]{eulervm}
\usepackage{amsmath,amssymb}
\usepackage{tensor}
\usepackage{array} % for better arrays (eg matrices) in maths
\usepackage{inconsolata} % typewriter

% section head style
\usepackage{sectsty}
\allsectionsfont{\normalfont\sffamily\bfseries}
% \renewcommand{\abstractnamefont}{\normalfont\sffamily\bfseries}
\renewcommand{\abstractname}{\normalfont\sffamily\bfseries Abstract}

% title style ... this throws weird error
% \usepackage{titling}
% \pretitle{\sffamily\bfseries{}\begin{center}\sffamily\bfseries\LARGE\selectfont}
% \posttitle{\sffamily\bfseries{}\par\end{center}}
% \preauthor{\sffamily{}\begin{center}\normalfont\sffamily\selectfont}
% \postauthor{\sffamily{}\par\end{center}}
% \predate{\sffamily{}\begin{center}\normalfont\sffamily\selectfont}
% \postdate{\sffamily{}\par\end{center}}
% instead use doconce subst command

% TOC style
% \renewcommand{\cftsectionfont}{\normalfont\sffamily}
\let\oldtoc\tableofcontents
\renewcommand{\tableofcontents}{\sffamily\oldtoc}

% headers/footers
\fancypagestyle{fancy1}{%
	\renewcommand{\headrulewidth}{0pt} % customise the layout...
	\lhead{}\chead{}\rhead{}
	\lfoot{}\cfoot{}\rfoot{p~\thepage{}}
}%
\fancypagestyle{fancy2}{%
	\renewcommand{\headrulewidth}{0pt} % customise the layout...
	\lhead{}\chead{}\rhead{\S\slshape\nouppercase\leftmark}
	\lfoot{}\cfoot{}\rfoot{p~\thepage{}}
}%
\pagestyle{fancy2}

% insert custom LaTeX commands...

\raggedbottom
\makeindex
\usepackage[totoc]{idxlayout}   % for index in the toc
\usepackage[nottoc]{tocbibind}  % for references/bibliography in the toc

%-------------------- end preamble ----------------------

\begin{document}

% matching end for #ifdef PREAMBLE
% #endif

\newcommand{\exercisesection}[1]{\subsection*{#1}}


% ------------------- main content ----------------------



% ----------------- title -------------------------

\title{\sffamily\bfseries{}The \emph{spin-transport} documentation}

% ----------------- author(s) -------------------------

\author{\sffamily{}Rico A.R. Picone\footnote{Email: \texttt{rpicone@stmartin.edu}. Department of Mechanical Engineering, Saint Martin's University.}}

% ----------------- end author(s) -------------------------

\date{\sffamily{}Aug 29, 2017}
\maketitle

\begin{abstract}
The \emph{spin-transport} software (\href{{https://github.com/ricopicone/spin-transport}}{GitHub}) is for the dynamic simulation of bulk spin transport---diffusion and separation---in solid media.
The project is open-source and still in development.
\end{abstract}

\tableofcontents


\vspace{1cm} % after toc





% !split
\section{Installation}
\label{section:installation}

% spin-transport

This repository contains the (developing) open-source code for simulating bulk spin transport&mdash;diffusion and separation&mdash;in solid media. Multi-spin-species and magnetic resonance simulations are in development.

This is a [Python](https://www.python.org/) and [FEniCS](https://fenicsproject.org/) project. FEniCS is used to numerically solve the spin transport governing partial differential equations.

End users of this project write Python code to interface with FEniCS.


One must first have a working installation of FEniCS.
This README assumes the use of [Docker](https://www.docker.com/) for installation, which is documented [here](http://fenics.readthedocs.io/projects/containers/en/latest/).

Then [clone](https://help.github.com/articles/cloning-a-repository/) this repository to the host machine.


The FEniCS docs have a section on [workflow](http://fenics.readthedocs.io/projects/containers/en/latest/work_flows.html).
There are many ways to instantiate these good practices, but if you're using a *nix system, the following may be the easiest.

With the cloned \texttt{spin-transport} repository as your working directory, create a link in your path to \textbf{spin-transport}'s \texttt{fenics} executable bash script.

```bash
ln fenics /usr/local/bin
```

Now a FEniCS Python script \texttt{foo.py} can be started with the command \texttt{fenics foo.py} \textbf{from the host} instead of manually starting it from a Docker container.
This has several advantages, including that there is no need to move scripts into the container and that the complicated syntax need not be remembered.


To verify that everything is installed correctly, run the Poisson equation demo \Verb!ft01_poisson.py! ([source](https://fenicsproject.org/pub/tutorial/html/._ftut1004.html)) in your container.

If you installed the \texttt{fenics} bash script per the instructions above, you can use the following command (working directory: \texttt{spin-transport}).

```console
$ fenics ft01_poisson.py
```

If everything is working fine, the output should look something like the following.

```console
$ fenics ft01_poisson.py
Calling DOLFIN just-in-time (JIT) compiler, this may take some time.
--- Instant: compiling ---
Calling FFC just-in-time (JIT) compiler, this may take some time.
Calling FFC just-in-time (JIT) compiler, this may take some time.
Calling FFC just-in-time (JIT) compiler, this may take some time.
Solving linear variational problem.
\emph{*} Warning: Degree of exact solution may be inadequate for accurate result in errornorm.
Calling FFC just-in-time (JIT) compiler, this may take some time.
Calling FFC just-in-time (JIT) compiler, this may take some time.
  Ignoring precision in integral metadata compiled using quadrature representation. Not implemented.
Calling FFC just-in-time (JIT) compiler, this may take some time.
Calling FFC just-in-time (JIT) compiler, this may take some time.
Calling FFC just-in-time (JIT) compiler, this may take some time.
error_L2  = 0.00823509807335
error_max = 1.33226762955e-15
```

The directory \texttt{spin-transport/poisson} should have been created and should contain two files: \texttt{solution.pvd} and \texttt{solution000000.vtu}.
These files contain the solution data.


This work is supported by a grant from the Army Research Office, Materials Science Division under grant proposal \textbf{Nanoscale Spin Hyperpolarization and Imaging} <!-- TODO: grant number -->
with PI [John Marohn, PhD](http://marohn.chem.cornell.edu/).


This project stems from a collaboration among three institutions:

\begin{description}
\item[%s]
  [Cornell University](http://www.cornell.edu/), 

\item[%s]
  [Saint Martin's University](https://www.stmartin.edu/), and the

\item[%s]
  [University of Washington](http://www.washington.edu/).
\end{description}

\noindent
The lead contributor to this project is [Rico Picone, PhD](http://ricopic.one) of Saint Martin's University, co-PI on the ARO grant.
Other contributors include [John Marohn, PhD](http://marohn.chem.cornell.edu/) (Cornell, PI), John A. Sidles, PhD (Washington), Joseph L. Garbini, PhD (Washington), and Corinne Isaac (Cornell).
\section{Short theoretical introduction}
\label{section:shorttheory}





\bibliographystyle{plain}
\bibliography{papers}


% ------------------- end of main content ---------------

% #ifdef PREAMBLE
\end{document}
% #endif

