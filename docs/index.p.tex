%%
%% Automatically generated file from DocOnce source
%% (https://github.com/hplgit/doconce/)
%%
%%
% #ifdef PTEX2TEX_EXPLANATION
%%
%% The file follows the ptex2tex extended LaTeX format, see
%% ptex2tex: http://code.google.com/p/ptex2tex/
%%
%% Run
%%      ptex2tex myfile
%% or
%%      doconce ptex2tex myfile
%%
%% to turn myfile.p.tex into an ordinary LaTeX file myfile.tex.
%% (The ptex2tex program: http://code.google.com/p/ptex2tex)
%% Many preprocess options can be added to ptex2tex or doconce ptex2tex
%%
%%      ptex2tex -DMINTED myfile
%%      doconce ptex2tex myfile envir=minted
%%
%% ptex2tex will typeset code environments according to a global or local
%% .ptex2tex.cfg configure file. doconce ptex2tex will typeset code
%% according to options on the command line (just type doconce ptex2tex to
%% see examples). If doconce ptex2tex has envir=minted, it enables the
%% minted style without needing -DMINTED.
% #endif

% #define PREAMBLE

% #ifdef PREAMBLE
%-------------------- begin preamble ----------------------

\documentclass[%
oneside,                 % oneside: electronic viewing, twoside: printing
final,                   % draft: marks overfull hboxes, figures with paths
10pt]{article}

\listfiles               %  print all files needed to compile this document

\usepackage{relsize,makeidx,color,setspace,amsmath,amsfonts,amssymb}
\usepackage[table]{xcolor}
\usepackage{bm,ltablex,microtype}

\usepackage[pdftex]{graphicx}

\usepackage{ptex2tex}
% #ifdef MINTED
\usepackage{minted}
\usemintedstyle{default}
% #endif
\usepackage{fancyvrb}

\usepackage[T1]{fontenc}
%\usepackage[latin1]{inputenc}
\usepackage{ucs}
\usepackage[utf8x]{inputenc}

\usepackage{lmodern}         % Latin Modern fonts derived from Computer Modern

% Hyperlinks in PDF:
\definecolor{linkcolor}{rgb}{0,0,0.4}
\usepackage{hyperref}
\hypersetup{
    breaklinks=true,
    colorlinks=true,
    linkcolor=linkcolor,
    urlcolor=linkcolor,
    citecolor=black,
    filecolor=black,
    %filecolor=blue,
    pdfmenubar=true,
    pdftoolbar=true,
    bookmarksdepth=3   % Uncomment (and tweak) for PDF bookmarks with more levels than the TOC
    }
%\hyperbaseurl{}   % hyperlinks are relative to this root

\setcounter{tocdepth}{2}  % levels in table of contents

% --- fancyhdr package for fancy headers ---
\usepackage{fancyhdr}
\fancyhf{} % sets both header and footer to nothing
\renewcommand{\headrulewidth}{0pt}
\fancyfoot[LE,RO]{\thepage}
% Ensure copyright on titlepage (article style) and chapter pages (book style)
\fancypagestyle{plain}{
  \fancyhf{}
  \fancyfoot[C]{{\footnotesize \copyright\ 2017, Rico A.R. Picone. Released under CC Attribution 4.0 license}}
%  \renewcommand{\footrulewidth}{0mm}
  \renewcommand{\headrulewidth}{0mm}
}
% Ensure copyright on titlepages with \thispagestyle{empty}
\fancypagestyle{empty}{
  \fancyhf{}
  \fancyfoot[C]{{\footnotesize \copyright\ 2017, Rico A.R. Picone. Released under CC Attribution 4.0 license}}
  \renewcommand{\footrulewidth}{0mm}
  \renewcommand{\headrulewidth}{0mm}
}

\pagestyle{fancy}


% prevent orhpans and widows
\clubpenalty = 10000
\widowpenalty = 10000

% --- end of standard preamble for documents ---


% insert custom LaTeX commands...

\raggedbottom
\makeindex
\usepackage[totoc]{idxlayout}   % for index in the toc
\usepackage[nottoc]{tocbibind}  % for references/bibliography in the toc

%-------------------- end preamble ----------------------

\begin{document}

% matching end for #ifdef PREAMBLE
% #endif

\newcommand{\exercisesection}[1]{\subsection*{#1}}


% ------------------- main content ----------------------



% ----------------- title -------------------------

\thispagestyle{empty}

\begin{center}
{\LARGE\bf
\begin{spacing}{1.25}
The \emph{spin-transport} documentation
\end{spacing}
}
\end{center}

% ----------------- author(s) -------------------------

\begin{center}
{\bf Rico A.R. Picone (\texttt{rpicone@stmartin.edu})}
\end{center}

    \begin{center}
% List of all institutions:
\centerline{{\small Department of Mechanical Engineering, Saint Martin's University}}
\end{center}
    
% ----------------- end author(s) -------------------------

% --- begin date ---
\begin{center}
Aug 30, 2017
\end{center}
% --- end date ---

\vspace{1cm}

\begin{abstract}
The \emph{spin-transport} software (\href{{https://github.com/ricopicone/spin-transport}}{GitHub}) is for the dynamic simulation of bulk spin transport---diffusion and separation---in solid media.
The project is open-source and still in development.
\end{abstract}

\tableofcontents


\vspace{1cm} % after toc





% !split
\section{Installation}
\label{section:installation}

\section{spin-transport: introduction}

This repository contains the (developing) open-source code for simulating bulk spin transport---diffusion and separation---in solid media. Multi-spin-species and magnetic resonance simulations are in development.

This is a \href{{https://www.python.org/}}{Python} and \href{{https://fenicsproject.org/}}{FEniCS} project. FEniCS is used to numerically solve the spin transport governing partial differential equations.

End users of this project write Python code to interface with FEniCS.

\subsection{Installation}

One must first have a working installation of FEniCS.
This README assumes the use of \href{{https://www.docker.com/}}{Docker} for installation, which is documented \href{{http://fenics.readthedocs.io/projects/containers/en/latest/}}{here}.

Then \href{{https://help.github.com/articles/cloning-a-repository/}}{clone} this repository to the host machine.

\subsection{Workflow}

The FEniCS docs have a section on \href{{http://fenics.readthedocs.io/projects/containers/en/latest/work_flows.html}}{workflow}.
There are many ways to instantiate these good practices, but if you're using a \*nix system, the following may be the easiest.

With the cloned \texttt{spin-transport} repository as your working directory, create a link in your path to \textbf{spin-transport}'s \texttt{fenics} executable bash script.

\bshcod
ln fenics /usr/local/bin
\eshcod

Now a FEniCS Python script \texttt{foo.py} can be started with the command \texttt{fenics foo.py} \textbf{from the host} instead of manually starting it from a Docker container.
This has several advantages, including that there is no need to move scripts into the container and that the complicated syntax need not be remembered.

\subsection{Testing the installation}

To verify that everything is installed correctly, run the Poisson equation demo \Verb!ft01_poisson.py! (\href{{https://fenicsproject.org/pub/tutorial/html/._ftut1004.html}}{source}) in your container.

If you installed the \texttt{fenics} bash script per the instructions above, you can use the following command (working directory: \texttt{spin-transport}).

\bshcod
$ fenics ft01_poisson.py
\eshcod

If everything is working fine, the output should look something like the following.

\bshcod
$ fenics ft01_poisson.py
Calling DOLFIN just-in-time (JIT) compiler, this may take some time.
--- Instant: compiling ---
Calling FFC just-in-time (JIT) compiler, this may take some time.
Calling FFC just-in-time (JIT) compiler, this may take some time.
Calling FFC just-in-time (JIT) compiler, this may take some time.
Solving linear variational problem.
*** Warning: Degree of exact solution may be inadequate for accurate result in errornorm.
Calling FFC just-in-time (JIT) compiler, this may take some time.
Calling FFC just-in-time (JIT) compiler, this may take some time.
  Ignoring precision in integral metadata compiled using quadrature representation. Not implemented.
Calling FFC just-in-time (JIT) compiler, this may take some time.
Calling FFC just-in-time (JIT) compiler, this may take some time.
Calling FFC just-in-time (JIT) compiler, this may take some time.
error_L2  = 0.00823509807335
error_max = 1.33226762955e-15
\eshcod

The directory \texttt{spin-transport/poisson} should have been created and should contain two files: \texttt{solution.pvd} and \texttt{solution000000.vtu}.
These files contain the solution data.

\subsection{Acknowledgement}

This work is supported by a grant from the Army Research Office, Materials Science Division under grant proposal \textbf{Nanoscale Spin Hyperpolarization and Imaging}
with PI \href{{http://marohn.chem.cornell.edu/}}{John Marohn, PhD}.

\subsection{Contributors}

This project stems from a collaboration among three institutions:

\begin{description}
\item[%s]
  [Cornell University](http://www.cornell.edu/),

\item[%s]
  [Saint Martin's University](https://www.stmartin.edu/), and the

\item[%s]
  [University of Washington](http://www.washington.edu/).
\end{description}

\noindent
The lead contributor to this project is \href{{http://ricopic.one}}{Rico Picone, PhD} of Saint Martin's University, co-PI on the ARO grant.
Other contributors include \href{{http://marohn.chem.cornell.edu/}}{John Marohn, PhD} (Cornell, PI), John A. Sidles, PhD (Washington), Joseph L. Garbini, PhD (Washington), and Corinne Isaac (Cornell).

\section{Short theoretical introduction}
\label{section:shorttheory}

Here's some code.

\bpycod
x = 5
if x > 3:
  print('big!')
\epycod



\bibliographystyle{plain}
\bibliography{papers}


% ------------------- end of main content ---------------

% #ifdef PREAMBLE
\end{document}
% #endif

